
%----------------------------------------------------------------------------------------
%	PACKAGES AND OTHER DOCUMENT CONFIGURATIONS
%----------------------------------------------------------------------------------------

\documentclass{article}

\input{structure.tex} % Include the file specifying the document structure and custom commands

%----------------------------------------------------------------------------------------
%	ASSIGNMENT INFORMATION
%----------------------------------------------------------------------------------------

\title{Αρχιτεκτονική Παράλληλων και Κατανεμημένων Υπολογιστών} % Title of the assignment

\author{Άρης Ζερβάκης - Στέφανος Καλογεράκης} % Author name and email address

\date{Πολυτεχνείο Κρήτης --- \today} % University, school and/or department name(s) and a date

%----------------------------------------------------------------------------------------

\begin{document}

\maketitle % Print the title

%----------------------------------------------------------------------------------------
%	INTRODUCTION
%----------------------------------------------------------------------------------------

\section*{Εισαγωγή} % Unnumbered section
Στη δεύτερη εργαστηριακή άσκηση καλεστήκαμε να χρησιμοποιήσουμε συνδιαστικά \eng{Streaming SIMD Extensions (SSE), MPI} και \eng{Pthreads} με σκοπό την παραλληλοποίηση του υπολογισμού του ω \eng{statistic}, το οποίο εφαρμόζεται για ανίχνευση θετικής επιλογής σε ακολουθίες \eng{DNA}. Ως κώδικας αναφοράς, χρησιμοποιήθηκε που δόθηκε απο τον διδάσκοντα του μαθήματος (φάκελος με όνομα \eng{Serial}).\\
\begin{warn}[] % Information block
	Για την υλοποίηση της άσκησης μας προχωρήσαμε σε 4 διαφορετικές υλοποιήσεις:\\
    1. Παραλληλοποίηση με \eng{SSE} Εντολές\\
    2. Παραλληλοποίηση με \eng{SSE} Εντολές και \eng{Pthreads}\\
    3. Παραλληλοποίηση με \eng{SSE} Εντολές και \eng{Pthreads} και \eng{MPI}\\
    4. (\eng{Bonus}) Παραλληλοποίηση με \eng{SSE} Εντολές για διαφορετικά \eng{Memory Layout}
\end{warn}

%----------------------------------------------------------------------------------------
%	Υλοποίηση
%----------------------------------------------------------------------------------------

\section{Υλοποίηση} % Numbered section
Για τον σειριακό υπολογισμό του ω \eng{statistic}, χρησιμοποιήσαμε αυτούσιο τον \eng{reference code}χωρίς να πραγματοποίησουμε αλλαγές και για αυτό δεν γίνεται κάποια παραπάνω αναφορά. Παρακάτω γίνεται ανάλυση όλων των μεθόδων παραλληλοποίησης που μελετήθηκαν στα πλαίσια αυτού του πρότζεκτ.\\

\subsection{\eng{SSE} Εντολές}

Η μέθοδος παραλληλοποίησης με τη χρήση \eng{SSE} εντολών υλοποιήθηκε με τη χρήση \eng{pointers} όπως είδαμε και στις διαλέξεις. Αξίζει να σχολιάσουμε ότι επιλέξαμε την συγκεκριμένη μέθοδο σε σύγκριση με την υλοποίηση με χρήση \eng{load} καθώς όπως διδαχτήκαμε είναι πιο γρήγορη. Μετά από δοκιμή στα δικά μας δεδομένα-υπολογισμούς επιβεβαιώσαμε το συγκεκριμένο γεγονός αφού η χρήση των \eng{pointers} οδήγησε σε λίγο πιο γρήγορο χρόνο εκτέλεσης που είναι και βασικό ζητούμενο της παραλληλοποίησης.\\
\newline
-Όλες οι μεταβλητές οι οποίες ξεκινούν με \eng{underscore} συσχετίζονται με το λειτουργικό κομμάτι της \eng{SSE} υλοποίησης.\\
-Δημιουργήθηκαν οι παρακάτω μεταβλητές πλάτους 128 \eng{bits}.
\selectlanguage{english}
% File contents

\begin{lstlisting}
 __m128 *mVec_ptr = (__m128 *) mVec;
 __m128 *nVec_ptr = (__m128 *) nVec;
 __m128 *LVec_ptr = (__m128 *) LVec;
 __m128 *RVec_ptr = (__m128 *) RVec;
 __m128 *CVec_ptr = (__m128 *) CVec;
 __m128 *FVec_ptr = (__m128 *) FVec;

 __m128 avgF_vec = _mm_setzero_ps();
 __m128 maxF_vec = _mm_setzero_ps();
 __m128 minF_vec = _mm_set_ps1(FLT_MAX);
\end{lstlisting}

\selectlanguage{greek}

Για την υλοποίηση των υπολογισμών έγιναν οι παρακάτω αλλαγές:\\
-Αλλαγή των \eng{malloc}, \eng{free} με τις \textunderscore \textunderscore \eng{mm}\textunderscore \eng{malloc}, \textunderscore \textunderscore \eng{mm}\textunderscore \eng{free}. 
(εντολές που χρησιμοποιούνται για ευθυγράμμιση των δεδομένων).\\
-Tροποποίηση εντολών.
(Σε σχόλια παρατίθενται οι εντολές στην αρχική μορφή τους και έπειτα η τροποποίηση τους)
\newline
\selectlanguage{english}
\begin{lstlisting}
for(unsigned int i=0; i<N/4 ;i++){

            //float num_0 = LVec[i] + RVec[i];
            temp_num_0 = _mm_add_ps(LVec_ptr[i],RVec_ptr[i]);

            //float num_1 = mVec[i]*(mVec[i]-1.0f)/2.0f;
            temp_num_1 = _mm_sub_ps(mVec_ptr[i], temp_one);
            temp_num_1 = _mm_mul_ps(mVec_ptr[i], temp_num_1);
            temp_num_1 = _mm_div_ps(temp_num_1, temp_two);

            //float num_2 = nVec[i]*(nVec[i]-1.0f)/2.0f;
            temp_num_2 = _mm_sub_ps(nVec_ptr[i], temp_one);
            temp_num_2 = _mm_mul_ps(nVec_ptr[i], temp_num_2);
            temp_num_2 = _mm_div_ps(temp_num_2, temp_two);

            //float num = num_0/(num_1+num_2);
            temp_num = _mm_add_ps(temp_num_1,temp_num_2);
            temp_num = _mm_div_ps(temp_num_0, temp_num);

            //float den_0 = CVec[i]-LVec[i]-RVec[i];
            temp_den_0 = _mm_sub_ps(CVec_ptr[i], LVec_ptr[i]);
            temp_den_0 = _mm_sub_ps(temp_den_0, RVec_ptr[i]);

            //float den_1 = mVec[i]*nVec[i];
            temp_den_1 = _mm_mul_ps(mVec_ptr[i], nVec_ptr[i]);

            //float den = den_0/den_1;
            temp_den = _mm_div_ps(temp_den_0, temp_den_1);

            //FVec[i] = num/(den+0.01f);
            FVec_ptr[i] = _mm_add_ps(temp_den, __temp_one);
            FVec_ptr[i] = _mm_div_ps(temp_num, FVec_ptr[i]);

            //maxF = FVec[i]>maxF?FVec[i]:maxF;
            maxF_vec = _mm_max_ps(FVec_ptr[i], maxF_vec);

            //minF = FVec[i]<minF?FVec[i]:minF;
            minF_vec = _mm_min_ps(FVec_ptr[i], minF_vec);

            //avgF += FVec[i];
            avgF_vec = _mm_add_ps(FVec_ptr[i], avgF_vec );
}
\end{lstlisting}
\selectlanguage{greek}
\vspace{5mm}
Στην υλοποίηση πραγματοποιούμε \eng{loop unrolling} και \eng{jamming} στις εντολές του \eng{for-loop}, με κάθε \eng{i} να αναλογεί σε 4 στοιχεία. ΄Οσον αφορα την εύρεση του \eng{max, min, avg} έχοντας ορίσει τις κατάλληλες \eng{m128} μεταβλητές πραγματοποιύμε επιμέρους σε συγκρίσεις ανά τετράδες με \eng{SSE} εντολές και στο τέλος αποθηκεύουμε σε μια \eng{global} μεταβλητή την σωστή τιμή ανάλογα με την περίπτωση.\\
\newpage
\selectlanguage{english}
\begin{lstlisting}
maxF = maxF_vec[0];
maxF = maxCalc(maxF_vec[1],maxF);
maxF = maxCalc(maxF_vec[2],maxF);
maxF = maxCalc(maxF_vec[3],maxF);

minF = minF_vec[0];
minF = minCalc(minF_vec[1],minF);
minF = minCalc(minF_vec[2],minF);
minF = minCalc(minF_vec[3],minF);

avgF = avgF_vec[0] + avgF_vec[1] + avgF_vec[2] + avgF_vec[3];
\end{lstlisting}
\selectlanguage{greek}
\vspace{5mm}
Στο τέλος, προσθέσαμε ένα κομμάτι κώδικα το οποίο για τα συγκεκριμένα δεδομένα που δοκιμάζουμε δεν πρόκειται να χρησιμοποιηθεί αλλά εισάγεται για λόγους πληρότητας, σε περίπτωση που υπάρξει ανάγκη για δοκιμή σε άλλα δεδομένα.

\selectlanguage{english}
\begin{lstlisting}
for (int j = (N - N % 4); j < N; j++) {
            float num_0 = LVec[j] + RVec[j];
            float num_1 = mVec[j] * (mVec[j] - 1.0f) / 2.0f;
            float num_2 = nVec[j] * (nVec[j] - 1.0f) / 2.0f;
            float num = num_0 / (num_1 + num_2);
            float den_0 = CVec[j] - LVec[j] - RVec[j];
            float den_1 = mVec[j] * nVec[j];
            float den = den_0 / den_1;

            FVec[j] = num / (den + 0.01f);
            maxF = FVec[j] > maxF ? FVec[j] : maxF;
            minF = FVec[j]<minF?FVec[j]:minF;
            avgF += FVec[j];
}
\end{lstlisting}
\selectlanguage{greek}
%------------------------------------------------

\subsection{\eng{SSE} Εντολές και \eng{Pthreads} }

Στο δέυτερο μέρος έπρεπε να παραλληλοποιήσουμε το \eng{reference code} συνδιαστικά, με \eng{SSE} Εντολές και \eng{Pthreads}. 
Τα \eng{Pthreads} δημιουργούνται στην αρχή της \eng{main} και γίνονται \eng{join} πριν την επιστροφή της. Στην υλοποίηση μας όσο το \eng{master thread} αρχικοποιεί τις μεταβλητές μας, τα \eng{worker threads} είναι σε κατάσταση \eng{busy wait} και έπειτα σε κάθε \eng{iteration} του \eng{for-loop} το \eng{master} μοιράζει στα \eng{worker threads} τους υπολογισμούς. Ο συγχρονισμός των \eng{Pthreads} σε κάθε \eng{iteration} επιτυγχάνεται με τη χρήση \eng{barrier}.


\subsection{\eng{SSE} Εντολές, \eng{Pthreads} και \eng{MPI} }

In malesuada ullamcorper urna, sed dapibus diam sollicitudin non. Donec elit odio, accumsan ac nisl a, tempor imperdiet eros. Donec porta tortor eu risus consequat, a pharetra tortor tristique. Morbi sit amet laoreet erat. Morbi et luctus diam, quis porta ipsum. Quisque libero dolor, suscipit id facilisis eget, sodales volutpat dolor. Nullam vulputate interdum aliquam. Mauris id convallis erat, ut vehicula neque. Sed auctor nibh et elit fringilla, nec ultricies dui sollicitudin. Vestibulum vestibulum luctus metus venenatis facilisis. Suspendisse iaculis augue at vehicula ornare. Sed vel eros ut velit fermentum porttitor sed sed massa. Fusce venenatis, metus a rutrum sagittis, enim ex maximus velit, id semper nisi velit eu purus.

\newpage
\subsection{\eng{BONUS}: Υλοποίηση διαφορετικών \eng{memory layout} για τη βελτιστοποίηση της παραλληλοποίησης με \eng{SSE} Εντολές }

In malesuada ullamcorper urna, sed dapibus diam sollicitudin non. Donec elit odio, accumsan ac nisl a, tempor imperdiet eros. Donec porta tortor eu risus consequat, a pharetra tortor tristique. Morbi sit amet laoreet erat. Morbi et luctus diam, quis porta ipsum. Quisque libero dolor, suscipit id facilisis eget, sodales volutpat dolor. Nullam vulputate interdum aliquam. Mauris id convallis erat, ut vehicula neque. Sed auctor nibh et elit fringilla, nec ultricies dui sollicitudin. Vestibulum vestibulum luctus metus venenatis facilisis. Suspendisse iaculis augue at vehicula ornare. Sed vel eros ut velit fermentum porttitor sed sed massa. Fusce venenatis, metus a rutrum sagittis, enim ex maximus velit, id semper nisi velit eu purus.

\begin{figure}[h!]
\centering
  \includegraphics[width=0.8\linewidth]{SCHEME.png}
  \caption{\eng{Comment}}
\end{figure}
%----------------------------------------------------------------------------------------
%----------------------------------------------------------------------------------------
\newpage
\section{Εκτέλεση}

Για την εκτέλεση, απλά καλούμαστε να τρέξουμε στο \eng{command line} την παρακάτω εντολή, παίρνοντας τα ακόλουθα
αποτελέσματα:
\vspace{10mm}
\selectlanguage{english}
% Command-line "screenshot"
\begin{commandline}
	\begin{verbatim}
$ ./run.sh

---------------------- Building everything ------------------------------


---------------------- Giving permissions -------------------------------


---------------------- Reference Code N:10000000-------------------------
Omega time 0.177147s - Total time 2.829882s - Min -2.711918e+06 - 
Max 6.048419e+05 - Avg -6.529043e-01



---------------------- SSE N:10000000------------------------------------
Omega time 0.106534s - Total time 2.149127s - Min -2.711918e+06 - 
Max 6.048419e+05 - Avg -6.348448e-01



-------------- SSE-PTHREADS N:10000000, PTHREADS:2-----------------------
Omega time 0.053595s - Total time 2.226116s - Min -2.711918e+06 - 
Max 6.048419e+05 - Avg -6.332115e-01


--------------- SSE-PTHREADS N:10000000, PTHREADS:4----------------------
Omega time 0.044613s - Total time 2.552894s - Min -2.711918e+06 - 
Max 6.048419e+05 - Avg -6.323332e-01
      \end{verbatim}
\end{commandline}


\begin{commandline}
      \begin{verbatim}


-------- SSE-PTHREADS-MPI N:10000000, PTHREADS:2, PROCESSES:2------------
Omega time 0.041026s - Total time 2.108915s - Min -2.711918e+06 - 
Max 6.048419e+05 - Avg -6.323332e-01


-------- SSE-PTHREADS-MPI N:10000000, PTHREADS:4, PROCESSES:2------------
Omega time 0.040200s - Total time 4.056555s - Min -2.711918e+06 - 
Max 6.048419e+05 - Avg -6.323262e-01


-------- SSE-PTHREADS-MPI N:10000000, PTHREADS:2, PROCESSES:4------------
Omega time 0.028752s - Total time 4.942250s - Min -2.711918e+06 - 
Max 6.048419e+05 - Avg -6.323262e-01


-------- SSE-PTHREADS-MPI N:10000000, PTHREADS:4, PROCESSES:4------------
Omega time 0.033914s - Total time 8.615657s - Min -2.711918e+06 - 
Max 6.048419e+05 - Avg -6.321101e-01


----------------------- Bonus -------------------------------------------


---------------------- SSE N:10000000(AGAIN)-----------------------------
Omega time 0.108413s - Total time 2.189199s - Min -2.711918e+06 - 
Max 6.048419e+05 - Avg -6.348448e-01


------------ SSE N:10000000, SSE_MEM_LAYOUT 3 vectors--------------------
Omega time 0.105147s - Total time 2.193838s - Min -2.711918e+06 - 
Max 6.048419e+05 - Avg -6.348448e-01


------------ SSE N:10000000, SSE_MEM_LAYOUT 2 vectors--------------------
Omega time 0.104894s - Total time 2.180221s - Min -2.711918e+06 - 
Max 6.048419e+05 - Avg -6.348448e-01


------------ SSE N:10000000, SSE_MEM_LAYOUT 1 vector --------------------
Omega time 0.104966s - Total time 2.203330s - Min -2.711918e+06 - 
Max 6.048419e+05 - Avg -6.348448e-01

      \end{verbatim}
\end{commandline}

\selectlanguage{greek}
% Warning text, with a custom title
\begin{warn}[Σημείωση:]
  Για να τρέξουμε το \eng{run script} είναι απαραίτητο να βρισκόμαστε στο \eng{directory} όπου έχουμε τοποθετήσει τα αρχεία μας και να έχουμε ήδη εγκατεστημένα το \eng{gcc, make} και \eng{mpich} ώστε να μπορεί να γίνει \eng{compile} και να πάρουμε τα αποτελέσματα.
\end{warn}

%----------------------------------------------------------------------------------------

\section{Συμπεράσματα}

Απεικονίζοντας τους παραπάνω χρόνους μπορούμε να εξάγουμε πολύτιμα συμπεράσματα για τις υλοποιήσεις μας, υπολογίζοντας το \eng{speedup} που έχουμε στην εκάστοτε περίπτωση.\\
Το \eng{speedup} υπολογίζεται με τον ακόλουθο τύπο:

\selectlanguage{english}
\begin{equation}
     Speedup = \dfrac{Serial Code Execution Time}{Parallelized Code Execution Time}
\end{equation}
\selectlanguage{greek}


\subsection{\eng{SSE} Εντολές}

In malesuada ullamcorper urna, sed dapibus diam sollicitudin non. Donec elit odio, accumsan ac nisl a, tempor imperdiet eros. Donec porta tortor eu risus consequat, a pharetra tortor tristique. Morbi sit amet laoreet erat. Morbi et luctus diam, quis porta ipsum. Quisque libero dolor, suscipit id facilisis eget, sodales volutpat dolor. Nullam vulputate interdum aliquam. Mauris id convallis erat, ut vehicula neque. Sed auctor nibh et elit fringilla, nec ultricies dui sollicitudin. Vestibulum vestibulum luctus metus venenatis facilisis. Suspendisse iaculis augue at vehicula ornare. Sed vel eros ut velit fermentum porttitor sed sed massa. Fusce venenatis, metus a rutrum sagittis, enim ex maximus velit, id semper nisi velit eu purus.

\begin{figure}[h!]
\centering
  \includegraphics[width=0.8\linewidth]{SSE.jpeg}
  \caption{\eng{Comparing Serial with SSE Implementation}}
\end{figure}

\newpage
\subsection{\eng{SSE} Εντολές, \eng{Pthreads}}

In malesuada ullamcorper urna, sed dapibus diam sollicitudin non. Donec elit odio, accumsan ac nisl a, tempor imperdiet eros. Donec porta tortor eu risus consequat, a pharetra tortor tristique. Morbi sit amet laoreet erat. Morbi et luctus diam, quis porta ipsum. Quisque libero dolor, suscipit id facilisis eget, sodales volutpat dolor. Nullam vulputate interdum aliquam. Mauris id convallis erat, ut vehicula neque. Sed auctor nibh et elit fringilla.

\begin{figure}[h!]
\centering
  \includegraphics[width=0.8\linewidth]{SSE_PTHREADS.jpeg}
  \caption{\eng{Comparing Serial with SSE and Pthreads Implementation}}
\end{figure}

\newpage
\subsection{\eng{SSE} Εντολές, \eng{Pthreads} και \eng{MPI} }

Στην περίπτωση παραλληλοποίησης του κώδικα με \eng{MPI} δε μας ζητήθηκε να αξιολογήσουμε την απόδοση και το \eng{speedup}. Ακολουθούν ενδεικτικά τα διαγράμματα με τους χρόνους που 
λάβαμε:
\vspace{5mm}
\begin{figure}[h!]
\centering 
  \includegraphics[width=0.8\linewidth]{MPI2.jpeg}
  \caption{\eng{Comparing Serial with SSE,Pthreads and MPI Implementation with 2 Processes} }
\end{figure}

\begin{figure}[h!]
\centering
  \includegraphics[width=0.8\linewidth]{MPI4.jpeg}
  \caption{\eng{Comparing Serial with SSE,Pthreads and MPI Implementation with 2 Processes} }
\end{figure}

\newpage

\subsection{\eng{BONUS}: \eng{SSE Memory Layouts}}
% Math equation/formula

In malesuada ullamcorper urna, sed dapibus diam sollicitudin non. Donec elit odio, accumsan ac nisl a, tempor imperdiet eros. Donec porta tortor eu risus consequat, a pharetra tortor tristique. Morbi sit amet laoreet erat. Morbi et luctus diam, quis porta ipsum. Quisque libero dolor, suscipit id facilisis eget, sodales volutpat dolor. Nullam vulputate interdum aliquam. Mauris id convallis erat, ut vehicula neque. Sed auctor nibh et elit fringilla, nec ultricies dui sollicitudin. Vestibulum vestibulum luctus metus venenatis facilisis. Suspendisse iaculis augue at vehicula ornare. Sed vel eros ut velit fermentum porttitor sed sed massa. Fusce venenatis, metus a rutrum sagittis, enim ex maximus velit, id semper nisi velit eu purus.

\begin{figure}[h!]
\centering 
  \includegraphics[width=0.8\linewidth]{BONUS.jpeg}
  \caption{\eng{Comparing different SSE Memory Layouts} }
\end{figure}

In malesuada ullamcorper urna, sed dapibus diam sollicitudin non. Donec elit odio, accumsan ac nisl a, tempor imperdiet eros. Donec porta tortor eu risus consequat, a pharetra tortor tristique. Morbi sit amet laoreet erat. Morbi et luctus diam, quis porta ipsum. Quisque libero dolor, suscipit id facilisis eget, sodales volutpat dolor. Nullam vulputate interdum aliquam. Mauris id convallis erat, ut vehicula neque. Sed auctor nibh et elit fringilla, nec ultricies dui sollicitudin. Vestibulum vestibulum luctus metus venenatis facilisis. Suspendisse iaculis augue at vehicula ornare. Sed vel eros ut velit fermentum porttitor sed sed massa. Fusce venenatis, metus a rutrum sagittis, enim ex maximus velit, id semper nisi velit eu purus.







%\begin{equation}
%     I = \int_{a}^{b} f(x) \; \text{d}x.
%\end{equation}
\end{document}
